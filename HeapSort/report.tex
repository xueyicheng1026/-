\documentclass[UTF8]{ctexart}
\usepackage{geometry, CJKutf8}
\geometry{margin=1.5cm, vmargin={0pt,1cm}}
\setlength{\topmargin}{-1cm}
\setlength{\paperheight}{29.7cm}
\setlength{\textheight}{25.3cm}
\usepackage{amsfonts}
\usepackage{amsmath}
\usepackage{amssymb}
\usepackage{amsthm}
\usepackage{enumerate}
\usepackage{graphicx}
\usepackage{multicol}
\usepackage{fancyhdr}
\usepackage{layout}
\usepackage{listings}
\usepackage{forest}
\usepackage{float, caption}
\usepackage{listings}
\usepackage{xcolor}
\usepackage{array}


\lstset{
    basicstyle=\ttfamily, basewidth=0.5em,
    language=C++,                    
    basicstyle=\ttfamily\small,      
    keywordstyle=\color{blue},       
    commentstyle=\color{green!50!black}, 
    stringstyle=\color{red},         
    numbers=left,                    
    numberstyle=\tiny\color{gray},   
    stepnumber=1,                    
    breaklines=true,                 
    frame=single,                    
    tabsize=4,                        
    showstringspaces=false,     
}

\newcommand{\dif}{\mathrm{d}}
\newcommand{\avg}[1]{\left\langle #1 \right\rangle}
\newcommand{\difFrac}[2]{\frac{\dif #1}{\dif #2}}
\newcommand{\pdfFrac}[2]{\frac{\partial #1}{\partial #2}}
\newcommand{\OFL}{\mathrm{OFL}}
\newcommand{\UFL}{\mathrm{UFL}}
\newcommand{\fl}{\mathrm{fl}}
\newcommand{\op}{\odot}
\newcommand{\Eabs}{E_{\mathrm{abs}}}
\newcommand{\Erel}{E_{\mathrm{rel}}}

\begin{document}

\pagestyle{fancy}
\fancyhead{}
\lhead{薛亦呈, 3230104027}
\chead{AvlTree remove函数}
\rhead{\today}

\section{程序设计思路}
声明:所建堆为最小堆

\subsection{升堆函数}
给定待操作元素的位置pos,与他的父亲节点比较大小,若待操作元更小则交换两者位置,while循环直到到达根节点或者待操作元比父亲节点大

\begin{lstlisting}
    void percolate_up(int pos) {
        while (pos > 1 && data[pos] < data[(pos)>>1])
        {
            swap(data[pos], data[pos>>1]);
            pos >>= 1;
        }
    }
\end{lstlisting}

\subsection{降堆函数}
给定待操作元素的位置pos,与他的两个子节点比较大小,若待操作元更大则与其中较小的一个(只有一个子节点并比他大时直接交换)交换位置,while循环直到到达叶子节点或者待操作元比所有子节点小

\begin{lstlisting}
    void percolate_down(int pos) {
        while (true) {
            int smallest = pos;
            int left = 2 * pos ;
            int right = 2 * pos + 1;
            // if (left < currentSize && (data)[left] < (data)[smallest])
            //     smallest = left;
            smallest = (left < currentSize && (data)[left] < (data)[smallest]) ? left : smallest;
            // if (right < currentSize && (data)[right] < (data)[smallest])
            //     smallest = right;
            smallest = (right < currentSize && (data)[right] < (data)[smallest]) ? right : smallest;
            if (smallest == pos) break;
            swap(data[pos], data[smallest]);
            pos = smallest;
        }   
    }

\end{lstlisting}

\subsection{建堆函数}
对一个已有的vector进行建堆,只需从叶子节点开始,从下往上对每个元素进行升堆操作,这样复杂度为O\text{(n)}

\begin{lstlisting}
    void build() {
        for (int i = currentSize>>1; i>=1; --i) {
            percolate_down(i);
        }
    }
\end{lstlisting}

\subsection{排序}
因为最小堆维护了根节点最小,所以每次只需要把根节点和最后一个节点交换位置,并让堆的大小-1(最后一个已经有序),然后对交换后的第一个元素进行一次降堆操作,维护堆的性质即可,总负责度为O\text{(nlogn)}

\begin{lstlisting}
    void sort() {
        int n = currentSize;
        for (int i = 1; i <= n; ++i){
            swap(data[1], data[currentSize]);
            --currentSize;
            percolate_down(1);
        }
        currentSize = n;
    }
\end{lstlisting}

\section{对比测试}

\begin{tabular}{|c|c|c|}
\hline
数据类型 & my\_heap\_sort & std::heap\_sort\\
\hline
Random sequence & 88ms & 48ms\\
\hline
Ordered sequence & 42ms & 32ms\\
\hline
Reverse sequence & 39ms &31ms\\
\hline
Repetitive sequence &77ms & 44ms\\
\hline

\end{tabular}

\section{额外测试}
    
\subsection{不开启O2优化}
在数据量较小\text{(<10000左右)}时,std::heap\_sort的效率更高,但是由于数据量非常小,只相差几毫秒可以忽略

在数据量较大\text{(>10000000)}时,my\_heap\_sort效率更高,数据量为1000000时误差在500毫秒左右

\subsection{开启O2优化}
无论数据量大小,std::heap\_sort的效率均更高,在标准数据量\text{(1000000)}时,std::heap\_sort效率更高,大约为1.7倍,但是当数据量不断升高时,两者效率越来越接近,在数据量为20000000时几乎持平,后续my\_heap\_sort效率更高,而当数据量非常大时\text{(以100000000为例)},my\_heap\_sort\text{(总耗时为18秒)}比std::heap\_sort快了10秒。

\section{原因分析}

std::heap\_sort可能在数据量较小时混合使用了其他排序方法

my\_heap\_sort内联小函数较多,不开启O2优化会有不必要开销

\section{相应小修改}

将if语句改成了位运算? :

\end{document}

