\documentclass[UTF8]{ctexart}
\usepackage{geometry, CJKutf8}
\geometry{margin=1.5cm, vmargin={0pt,1cm}}
\setlength{\topmargin}{-1cm}
\setlength{\paperheight}{29.7cm}
\setlength{\textheight}{25.3cm}
\usepackage{amsmath, amssymb, amsfonts, amsthm}
\usepackage{enumerate}
\usepackage{graphicx}
\usepackage{fancyhdr}
\usepackage{graphicx}
\usepackage{multicol}
\usepackage{xcolor}
\usepackage{layout}
\usepackage{listings}
\usepackage{array}
\usepackage{algorithm}
\usepackage{algpseudocode}

% 设置代码高亮样式
\lstset{
    basicstyle=\ttfamily\large,
    numbers=none,
    numberstyle=\tiny,
    stepnumber=1,
    numbersep=5pt,
    backgroundcolor=\color{gray!10},
    showspaces=false,
    showstringspaces=false,
    showtabs=false,
    frame=single,
    rulecolor=\color{black},
    tabsize=2,
    captionpos=b,
    breaklines=true,
    breakatwhitespace=false,
    keywordstyle=\color{blue},
    commentstyle=\color{green!60!black},
    stringstyle=\color{red!80!black},
    escapeinside={(*@}{@*)} % 支持插入 LaTeX 数学符号
}

% 自定义命令
\newcommand{\dif}{\mathrm{d}}
\newcommand{\avg}[1]{\left\langle #1 \right\rangle}
\newcommand{\difFrac}[2]{\frac{\dif #1}{\dif #2}}
\newcommand{\pdfFrac}[2]{\frac{\partial #1}{\partial #2}}

% 设置页眉
\pagestyle{fancy}
\fancyhead{}
\fancyhead[L]{薛亦呈, 3230104027}
\fancyhead[C]{LIS}
\fancyhead[R]{\today}

\begin{document}

\section{初始思路}

\subsection{数组定义}
假设 \texttt{num} 为一维线性数组,存放总序列,\texttt{size\_of\_num = n}。

定义一维数组 \texttt{dp},第 $i$ 个位置用于储存 \texttt{num[1:i]} 的 LIS 长度。

\subsection{状态转移方程}
\texttt{dp[i]} 应该从前 $i-1$ 个状态进行更新,遍历前 $i-1$ 个元素(下标用 $k$ 表示)。  
如果发现 \texttt{num[i] > num[k]},那么可以构成一个递增数列,对 \texttt{dp[i]} 进行更新,每次更新时保留最大值,即:
\[
\texttt{dp[i]} = \max\{\texttt{dp[i]},\texttt{dp[k]} + 1\}
\]

\subsection{伪代码}
\begin{lstlisting}[caption={O(N\textsuperscript{2}) 解法}]
SET n = size_of_num
FOR i = 1 TO n DO
    dp[i] = 1  // 初始化,一个独立数字可以构成长度为1的递增序列
FOR i = 1 TO n DO
    FOR k = 1 TO i-1 DO
        IF num[i] > num[k] THEN
            dp[i] = MAX(dp[i], dp[k] + 1)
result = dp[n,n]
PRINT(result)
\end{lstlisting}

\subsection{缺陷与修改思路}
\noindent -该算法时间复杂度为 $O(N^2)$。

\noindent -分析发现,对前 $i-1$ 个 \texttt{num} 进行遍历时使用了比较大小操作,是否可以通过让前 $i-1$ 个 \texttt{num} 有序,并使用二分查找优化到 $O(N\log N)$?

\section{改进思路}
分析发现,更新递增子序列长度,需要比较原有递增子序列的尾数大小,若有两个原有递增子序列,它们的长度都相同,那么可以“贪心地”只考虑尾数更小的那一个
例如:\texttt{7, 2, 10, 9, 3, 4}。  
遍历到 \texttt{4} 时,前面原有的最长递增子序列有多个:
\texttt{7, 10; 7, 9; 2, 10; 2, 9; 2, 3}。  
此时只需保留尾数最小的 \texttt{2, 3} 即可。

\subsection{优化方案}
\noindent \textbf{定义:}  
一维数组 \texttt{tails},第 $i$ 位储存当前指针之前,长度为 $i$ 的所有递增子序列尾数的最小值。  

\noindent \textbf{状态转移:}  
对于遍历指针所指向的数,如果它比 \texttt{tails} 中所有的数都大,则直接加到 \texttt{tails} 的末尾;否则,用它替换 \texttt{tails} 中第一个大于它的数。  

这样可以保持 \texttt{tails} 的升序性质,每次查找时可以使用二分查找。

\subsection{伪代码}
\begin{lstlisting}[caption={O(NlogN) 解法}]
SET n = size_of_num
DECLARE tails = []  // 初始化空数组
FOR i = 1 TO n Do
    // 二分查找 tails 中第一个比 num[i] 大的数
    pos = BinarySearch(tails, num[i])  
    IF pos == len(tails) Then
        tails.append(num[i])  // 若所有数都比 num[i] 小,将其加入
    ELSE
        tails[pos] = num[i]  // 替换更大的值
result = len(tails)
PRINT(result)
\end{lstlisting}

\end{document}