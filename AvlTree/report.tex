\documentclass[UTF8]{ctexart}
\usepackage{geometry, CJKutf8}
\geometry{margin=1.5cm, vmargin={0pt,1cm}}
\setlength{\topmargin}{-1cm}
\setlength{\paperheight}{29.7cm}
\setlength{\textheight}{25.3cm}
\usepackage{amsfonts}
\usepackage{amsmath}
\usepackage{amssymb}
\usepackage{amsthm}
\usepackage{enumerate}
\usepackage{graphicx}
\usepackage{multicol}
\usepackage{fancyhdr}
\usepackage{layout}
\usepackage{listings}
\usepackage{forest}
\usepackage{float, caption}


\lstset{
    basicstyle=\ttfamily, basewidth=0.5em
}

\newcommand{\dif}{\mathrm{d}}
\newcommand{\avg}[1]{\left\langle #1 \right\rangle}
\newcommand{\difFrac}[2]{\frac{\dif #1}{\dif #2}}
\newcommand{\pdfFrac}[2]{\frac{\partial #1}{\partial #2}}
\newcommand{\OFL}{\mathrm{OFL}}
\newcommand{\UFL}{\mathrm{UFL}}
\newcommand{\fl}{\mathrm{fl}}
\newcommand{\op}{\odot}
\newcommand{\Eabs}{E_{\mathrm{abs}}}
\newcommand{\Erel}{E_{\mathrm{rel}}}

\begin{document}

\pagestyle{fancy}
\fancyhead{}
\lhead{薛亦呈, 3230104027}
\chead{AvlTree remove函数}
\rhead{\today}

\section{程序设计思路}
首先,增加了balance函数,用于平衡某一节点的子树高度,判断该节点对应的四种子树高度情况,然后进行对应的rotation操作。

remove函数在BinarySearchTree的基础上,在每次递归调用返回时,对当前t指针指向节点调用balance函数,因为递归返回时是自下而上的,所以每次调用时,下方时保持平衡的,这样可以保持AVL树的平衡,并且detachMin函数中的删除操作有一种情况也会导致高度的失衡,因此在每次返回时也调用了balance函数。

顺便在insert函数里也添加了balance。

\section{测试时的问题}
当输入数据较大时,remove函数的递归栈内存空间会溢出(段错误),因此修改了一个迭代版本的remove,递归的版本命名成了BST\_0.h,对应的test函数命名为test\_0.cpp,可以用make old指令调用。

\end{document}

