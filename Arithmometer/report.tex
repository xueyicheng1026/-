\documentclass[UTF8]{ctexart}
\usepackage{geometry, CJKutf8}
\geometry{margin=1.5cm, vmargin={0pt,1cm}}
\setlength{\topmargin}{-1cm}
\setlength{\paperheight}{29.7cm}
\setlength{\textheight}{25.3cm}
\usepackage{amsfonts}
\usepackage{amsmath}
\usepackage{amssymb}
\usepackage{amsthm}
\usepackage{enumerate}
\usepackage{graphicx}
\usepackage{multicol}
\usepackage{fancyhdr}
\usepackage{layout}
\usepackage{listings}
\usepackage{forest}
\usepackage{float, caption}


\lstset{
    basicstyle=\ttfamily, basewidth=0.5em
}

\newcommand{\dif}{\mathrm{d}}
\newcommand{\avg}[1]{\left\langle #1 \right\rangle}
\newcommand{\difFrac}[2]{\frac{\dif #1}{\dif #2}}
\newcommand{\pdfFrac}[2]{\frac{\partial #1}{\partial #2}}
\newcommand{\OFL}{\mathrm{OFL}}
\newcommand{\UFL}{\mathrm{UFL}}
\newcommand{\fl}{\mathrm{fl}}
\newcommand{\op}{\odot}
\newcommand{\Eabs}{E_{\mathrm{abs}}}
\newcommand{\Erel}{E_{\mathrm{rel}}}

\begin{document}

\pagestyle{fancy}
\fancyhead{}
\lhead{薛亦呈, 3230104027}
\chead{四则运算器}
\rhead{\today}

\section{程序设计思路}
\subsection{实现功能}
1.常规四则运算

2.乘方\textasciicircum(连用时,优先级为从前向后)(附加)

3.负数的输入以及负号的叠用(附加)

4.小数的输入以及整数为0省略的情况

5.科学计数法的输入(附加)

6.判断错误类型(附加)

\subsection{合法输入}
小数(整数位是0可以省略,如.15表示0.15);科学计数法;若干个负号和其他运算符的连接(如1+---1、(--2+1)/---1))

\subsection{不合法输入}
1.括号不匹配

提示:Invalid expression: Bracket missmatch!

2.不正确的运算符、或非法字符:

提示:Invalid expression: Invalid operator!

3.运算符连续、运算符多余对于负号,数字或左括号前添加若干连续负号属于合法,但连续负号之间出现其他运算符属于错误,如:1*--2合法,1-*--2不合法,1--*2不合法

提示:Invalid expression: Operator used incorrectly!

4.除数为零

提示:Divided by zero!

5.科学计数法指数、底数缺失

提示:Invalid expression: Scientific notation error!

6.小数点误用

提示:Invalid expression: Decimal point error!

\subsection{预处理}
\subsubsection{负号合并}
将连续的奇数个负号合并为一个负号,偶数个合并为两个负号

\subsubsection{负号替换}
case1: 前一项为数字或')'的负号转变成"+(0-1)@":

\begin{center}
 			‘-’   --->   "+(0-1)@"
\end{center}

case2: 前一项为运算符或“(”的负号或第一个负号转变成"(0-1)@":

\begin{center}
			 ‘-’   --->   "(0-1)@"
\end{center}

其中@表示乘法,但是与普通乘法优先级不同,为优先级最高的运算

同时兼容了科学计数法输入时,指数为负数的情况,如:1e-1

这样支持了负数的输入,且保证了正常情况下不会出现两个连续运算符

\subsubsection{添加空格}
向所有运算符前后各增添一个空格,便于后续读取处理

\subsection{计算}
使用c++自带的sstream处理输入流
分别建立数字栈和运算符栈

遇到数字,直接压入数字栈
遇到运算符,若是空栈,直接入栈;若是左括号直接入栈;若是右括号进行弹出,直到遇到左括号;若是普通的四则运算符,若优先级>栈顶元素优先级,入栈;否则不断进行运算,直到栈顶元素小于当前运算符优先级,然后再入栈。

在读取完expression后,再执行栈中所有的运算,最终数字栈中的栈顶元素即表达式的值。
\subsection{非法判断}
在执行运算时,判断除数是否为0

运算前判断括号是否匹配

运算时若出现栈空的情况,则是运算符连续

\section{测试程序}
\subsection{非法判断测试}
\subsubsection{括号不匹配}

(1)1+2*(3-(4*2)))/2

(2)1+2*((3-(4*2))/2

\subsubsection{运算符使用错误}

(1)(-)

(2)1+2*3-4/+2

(3)1+2*3-4/2-
		               
(4)1+2--*3-4/2

(5)1@2

(6)2=1+1

\subsubsection{科学计数法指数或底数缺失}

(1)1.2E

(2)e-1

\subsubsection{被0除}

(1)1+2*3-4/0+2

\subsubsection{小数点误用测试}

(1)1.2.3+1

(2)..2*1

\subsection{负号处理测试}
(1)-1+-(1*-2)--5 等价于 -1-(1*(-2))+5

(2)1+2*--3-4/2 等价于 1+2*(-3)-4/2

\subsection{小数测试}
(1)1-.1/.01*(.2+.1) 等价于 1-0.1/0.01*(0.2+0.1)

(2)1-0.1/.01*(.2+.1) 等价于 1-0.1/0.01*(0.2+0.1)

(3)1-.1/-.01*(.2+-.1) 等价于 1-0.1/(-0.01)*(0.2-0.1)


\subsection{科学计数法测试}
(1)1.0e-2/1e-1-1E-1

(2)1.0e---2/1e-1-1E-1

\subsection{乘方测试}
(1)2\textasciicircum2+1

(2)2\textasciicircum-2+1

(3)2\textasciicircum--2+1

(4)2\textasciicircum(-2+1)

(5)2\textasciicircum-(-2+1)

(6)2\textasciicircum3\textasciicircum2

\subsection{混合运算测试}
(1)1024/2\textasciicircum e10--8

(2).125/8\textasciicircum(1+2/2---3)

(3)11e---2+.1+100\textasciicircum.5

(4)11e---2+.1+100\textasciicircum-.5


\end{document}

