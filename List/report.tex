\documentclass[UTF8]{ctexart}
\usepackage{geometry}  
\geometry{margin=1.5cm, vmargin={0pt,1cm}}  
\setlength{\topmargin}{-1cm}  
\setlength{\paperheight}{29.7cm}  
\setlength{\textheight}{25.3cm}  
  
% useful packages.  
\usepackage{amsfonts, amsmath, amssymb, amsthm}  
\usepackage{enumerate, graphicx, multicol, fancyhdr, layout}  
\usepackage{listings, float, caption}  
\usepackage{verbatim}  
  
\lstset{  
    basicstyle=\ttfamily, basewidth=0.5em  
}  
  
% some common command  
\newcommand{\dif}{\mathrm{d}}  
\newcommand{\avg}[1]{\left\langle #1 \right\rangle}  
\newcommand{\difFrac}[2]{\frac{\dif #1}{\dif #2}}  
\newcommand{\pdfFrac}[2]{\frac{\partial #1}{\partial #2}}  
\newcommand{\OFL}{\mathrm{OFL}}  
\newcommand{\UFL}{\mathrm{UFL}}  
\newcommand{\fl}{\mathrm{fl}}  
\newcommand{\op}{\odot}  
\newcommand{\Eabs}{E_{\mathrm{abs}}}  
\newcommand{\Erel}{E_{\mathrm{rel}}}  
  
\begin{document}  
  
\pagestyle{fancy}  
\fancyhead{}  
\lhead{薛亦呈, 3230104027}  
\chead{数据结构与算法第四次作业}  
\rhead{Oct.19th, 2024}  
  
\section{测试程序的设计思路}  
  
通过以下代码创建了 \texttt{lst1}, \texttt{lst2}, \texttt{lst3}, \texttt{lst4}, \texttt{lst5} 来测试 \texttt{List} 的构造函数和赋值运算。  
  
\begin{lstlisting}[language=C++]  
List<int> lst1;  
List<int> lst2 {1,3,5,7,9};  
List<int> lst3 = lst2;  
List<int> lst4 = List<int>({2,4,6,8,10});  
const List<int> lst5(lst3);  
\end{lstlisting}  
  
通过以下代码测试 \texttt{push\_back} 和 \texttt{push\_front} 的调用:  
  
\begin{lstlisting}[language=C++]  
for (int i = 0; i < 10; ++i) {  
    if (i % 2 == 0)   
        lst1.push_back(i);  
    else   
        lst1.push_front(i);  
}  
lst1.push_back(13);  
\end{lstlisting}  
  
测试结果如下:  
  
\begin{verbatim}  
9 7 5 3 1 0 2 4 6 8 13  
1 3 5 7 9  
1 3 5 7 9  
2 4 6 8 10  
1 3 5 7 9  
\end{verbatim}  
  
\texttt{size()} 的测试,正确输出为:  
  
\begin{verbatim}  
11  
\end{verbatim}  
  
\texttt{clear()} 和 \texttt{empty()} 的测试,正确输出为:  
  
\begin{verbatim}  
1  
\end{verbatim}  
  
\texttt{front()} 和 \texttt{back()} 的测试:  
  
\begin{verbatim}  
9  
13  
\end{verbatim}  
  
\texttt{pop\_front()} 和 \texttt{pop\_back()} 的测试:  
  
\begin{verbatim}  
100 7 5 3 1 0 2 4 6 8 200  
\end{verbatim}  
  
\texttt{erase()} 的测试:  
  
\begin{verbatim}  
7 5 3 1 0 2 4 6 8  
5  
5 3 1 0 2 4 6 8  
\end{verbatim}  
  
\texttt{iterator} 和 \texttt{const\_iterator} 的测试:  
  
\begin{verbatim}  
1 3 5 7 9  
1  
5  
5  
1  
\end{verbatim}  
  
\section{测试的结果}  
  
测试结果一切正常,完全符合预期输出。我用 Valgrind 进行测试,未发现内存泄露。  
  
\section{Bug 报告}  
  
\texttt{end()} 函数指向尾部哨兵,并不需要动态版本。  
  
\end{document}  